%% -*- coding: utf-8; -*-

% Use 'digital' option to enable back references. This option is recommended for digital pdf version
%\documentclass[phd,american,digital]{thesispuc}%english thesis
\documentclass[mscr,american]{thesispuc}%english dissertation
%\documentclass[phd,brazilian]{thesispuc}%tese em portugês
%\documentclass[msc,brazilian]{thesispuc}%disseretação em portuguŝ


%%%
%%% Additional Packages
%%%
\usepackage{tabularx}
\usepackage{multirow}
\usepackage{multicol}
\usepackage{colortbl}
\usepackage[%
    dvipsnames,
    svgnames,
    x11names,
    fixpdftex,
    table
]{xcolor}
\usepackage{numprint}
\usepackage{textcomp}
\usepackage{booktabs}
\usepackage{amsmath}
\usepackage{enumitem}
\usepackage{amssymb}
% ABNT reference style package. The current style is the alphabetical order, if you need
% change to citation order, change in the line above 'alf' to 'num', also at the end replace
% bibliographystyle with the commented version.
\usepackage[alf,bibjustif,abnt-emphasize=bf]{abntex2cite}
%\usepackage{tikz}
%\usepackage[linesnumbered, ruled, vlined]{algorithm2e}
%\usepackage{pgfplots,pgfplotstable} 
%\usepackage{array}

%% numprint 
\npthousandsep{.}
\npdecimalsign{,}

%% ThesisPUC option
%\tablesmode{fig} %% [nada, fig, tab ou figtab]
%\algoritmsmode{none} %% [none ou use] %% Default is [use]
%\codesmode{none} %% [none ou use] %% Default is [use]
%\abreviationsmode{none} %% [none ou use] %% Default is [use]

% \makeatletter  \renewcommand\@biblabel[1]{#1}  \makeatother

%%%
%%% Counters
%%%

%% uncomment and change for other depth values
\setcounter{tocdepth}{1}
%\setcounter{lofdepth}{3}
%\setcounter{lotdepth}{3}
%\setcounter{secnumdepth}{3}

%%%
%%% Misc.
%%%

\usecolour{true}


%%%
%%% Titulos
%%%

\author{Felipe V. Côrtes}
\authorR{Cortes, Felipe} % full name

\advisor{Roberto Ierusalimschy}{Prof.} %Name LastName
\advisorR{Ierusalimschy, Roberto} %LastName, Name
% If the advisor's department is different from author's department, uncomment the next line and type the correct name and acronym of advisor's institution.
%\advisorInst{institution name}{acronym}

%\coadvisor{Otávio da Fonseca Martins Gomes}{Dr.}
%\coadvisorR{da Fonseca Martins Gomes, Otávio}
%\coadvisorInst{Centro de Tecnologia Mineral}{CETEM/MCTI}

\title{Extrator de Tipos para Lua} %title in portuguese

\titleuk{Extrator de Tipos para Lua} %title in english

%%\subtitulo{Aqui vai o subtitulo caso precise}

\day{3}
\month{February}
\year{2023}

\city{Rio de Janeiro}
\CDD{004} 
\department{Informática}
\program{Informática}
\school{Pós-Graduação em Informática}
\university{Pontifícia Universidade Católica do Rio de Janeiro}
\uni{PUC-Rio }


% %%%
% %%% Jury
% %%%

% \jury{%
%   \jurymember{Alberto Barbosa Raposo}{Prof.}
%     {Departamento de Informática}{PUC-Rio}
%   \jurymember{Waldemar Celes Filho}{Prof.}
%     {Departamento de Informática}{PUC-Rio}
% }


% %%%
% %%% Personal Resume
% %%%

% \resume{%
% % If it fit in one line use this command:
% \makebox[\textwidth][s]{Graduated in computer science by the  Harvard University.}%
% % If not just type your resume without any special command 
% }

% %%%
% %%% Acknowledgment (REMINDER TO SCHOLARSHIP STUDENTS. Do not forget to thank the agencies that supported your work.)
% %%%
% \acknowledgment{%
% \noindent To my adviser Professor Marcelo Gattass for the stimulus and partnership
% to carry out this work.
% \bigskip

% \noindent To CNPq and PUC-Rio, for the aids granted, without which this work does not
% could have been accomplished.
% \bigskip

% \noindent \textbf{For students contemplated with any CAPES scholarship, whose defense occurred as of 04 September 2018 leave the following passage:} 

% \noindent This study was financed in part by the Coordenação de Aperfeiçoamento de Pessoal
% de Nível Superior - Brasil (CAPES) - Finance Code 001.
% }


%%%
%%% Catalog prekeywords
%%%

\catalogprekeywords{%
  \catalogprekey{Informática}%
}

%%%
%%% Keywords - Don't use % at the end of /key dfinition
%%%

\keywords{%
  \key{Lua}
  \key{Language}
  \key{Type}
  \key{Inspection}
}

\keywordsuk{%
  \key{Lua}
  \key{Language}
  \key{Type}
  \key{Inspection}
}

%%%
%%% Abstract
%%%

\abstract{%
Inspecionar código dinamicamente tipado é uma tarefa difícil devido a falta de informação sobre tipos. Inspirado por esse desafio, construímos uma ferramenta capaz de inspecionar programas feitos em Lua, extraindo os tipos das funções nele presentes. Sustentado pelas capacidades introspectivas da linguagem, é possível extrair os valores dos parâmetros e retornos de cada execução de função, gerando um relatório útil para documentação e inspeção. Esse documento apresenta a especificação e implementação do software, assim como os resultados obtidos em alguns programas Lua de referência.
}

\abstractuk{%
  Inspecting dynamically typed code is hard due to the lack of type information provided. Inspired by this challenge, we built a tool capable of inspecting Lua programs and extracting types from the functions in it. Sustained by the reflection capabilities of the language, it's possible to extract parameter and return values from each function execution and generate a useful report for code documentation and inspection. This document presents the software design and implementation, as well as results obtained by some Lua benchmark programs.
}

%%%
%%% Dedication
%%%

\dedication{%
  To my parents, for their support\\
and encouragement.
}

%%%
%%% Epigraph
%%%

% \epigraph{%
%   My beautifull epigraph
% }
% \epigraphauthor{Wassily Kandinsky}
% \epigraphbook{Regards sur le passé}


\begin{document}
  % -*- coding: utf-8; -*-

\chapter{Introduction}
There are several reasons that motivate the adoption of statically typed languages. Maintaining large systems built with dynamic types can become a nightmare due to the lack of type information \cite{takikawa_is_2016}. Typed languages also generally has better performance because compile-time type information helps generating optimized machine code. However, programmers are frequently left empty-handed when inspecting dynamically typed code while having to re-write systems to a statically typed languaged if gradually typed languages are not an option.
\paragraph*{}
Inspired by the challenge of inspecting dynamically typed code, we built a type extractor for the Lua programming language. By inspecting a program's execution during runtime, it can generate enough information to help programmers visualize the types being transfered between functions of their program. The software output can be used as an useful documentation, while also helping programmers migrate code to a statically typed one or even for debugging.
\paragraph*{}
The document is structured as follows. In Chapter ~\ref{cha:Previous Work} we present previous work related to type systems in Lua. In Chapter ~\ref{cha:Project Scope} we describe the software goal. Chapter ~\ref{cha:Project Specification} explain the software modules and how they interact. In Chapter ~\ref{cha:Development} it's shown the software key functions, the modules relationship and basic utilization. In Chapter ~\ref{cha:Results} we present and discuss some results obtained by the type extractor on some Lua benchmarks. Finally on Chapter ~\ref{cha:Final Considerations} we present our conclusion and future work.

% \begin{figure}
% \centering
% \includegraphics[width=0.45\textwidth]{pictures/image01.png}
% \includegraphics[width=0.45\textwidth]{pictures/image02.png}
% \caption{Meshes generated from medical data. Data obtained from the AIM$@$SHAPE Shape Repository \cite{AIMSHAPE}}
% \label{fig:example}
% \end{figure}

%This document is structured as follows. In Chapter~\ref{cha:Previous Work} we present some previous work relevant to our problem. In Chapter~\ref{cha:Proposal} we explain our proposal. In Chapter~\ref{cha:Results} we show our results. Finally, in Chapter~\ref{cha:Conclusion} we present our conclusion and future work.


 % introduction
  % -*- coding: utf-8; -*-

\chapter{Previous Work}
\label{cha:Previous Work}

There has been some notable works about Lua type system that we must cite. Typed Lua \cite{murbach_typed_2014} has already defined an optional type system for the language. More than enriching documentation, this extension ensures static type safety while preserving Lua idioms. Typed Lua encodes the main data structure mechanism from Lua into arrays, records, tuples and maps. It uses a bracket syntax to denote table types:

\lstinputlisting[label=insert,title={List Insert},caption={Insert Typed Lua}, language={[5.0]Lua}]{codes/typedlua.lua}

The type system is designed to be lightweight and type-safe and extends for typing object, classes and modules by adding type annotations. In Code~\ref{insert} example, a simple algorithm for inserting numbers in a list is shown using type annotations. The Element interface is defined recursively and referenced twice on the function's header, indicating it's return type. The \textit{?} symbol means that \textit{e} is optional and can assume empty values. Although Typed Lua's type system share some parts with other optional type systems for dynamically typed languages, it's design demanded uncommon features due to Lua's characteristics.
\par
Lua Type System has also been explored for scripting optimization with Pallene \cite{gualandi_pallene_2020}. The language design is inspired by optional type systems and it's semantical and syntatical similarity with Lua enables integrating seamlessly with Lua's dynamic code.
\clearpage

\lstinputlisting[label=sum,title={Array Sum},caption={Pallene Array Sum},language={[5.0]Lua}]{codes/pallene.lua}

As opposed to Typed Lua, Pallene is designed for efficiency. It performs runtime checks to ensure type safety with a particular flexibility. Similarly, Pallene uses type annotations. As shown in Code~\ref{sum}, the function \textit{sum} receives an array of float and returns a single float. Pallene has a built-in interoperability with Lua by sharing its runtime and data-structures. These features allow converting Lua code to Pallene code more easly. 

% Early smoothing methods tried to minimize... In the figure \ref{subfig:pictures/image01.png} we see...

% \subimages{A set of three subfigures:
% (a) describes the first subfigure;
% (b) describes the second subfigure;
% (c) describes the third subfigure.}{55}
% {
%  \subimage[Bamboo-pile Vertically Inserted Position]{.45}{pictures/image01.png}
%  \subimage[Bamboo-pile Normal Inserted Position]{.45}{pictures/image02.png}\\
%  \subimage[bamboo-pile Inserted 45° angle]{.45}{example-image}
% }
% \newpage
% \csubimages{A set of six subfigures in two pages.}{55}
% {
%  \subimage[Bamboo-pile Vertically Inserted Position]{.45}{pictures/image01.png}
%  \subimage[Bamboo-pile Normal Inserted Position]{.45}{pictures/image02.png}\\
%  \subimage[bamboo-pile Inserted 45° angle]{.45}{example-image}
% }
% \ssubimages{A set of six subfigures in two pages.(Continuation)}{55}
% {
%  \subimage[Bamboo-pile Vertically Inserted Position]{.45}{pictures/image01.png}
%  \subimage[Bamboo-pile Normal Inserted Position]{.45}{pictures/image02.png}\\
%  \subimage[bamboo-pile Inserted 45° angle]{.45}{example-image}
% } % previous work
  % -*- coding: utf-8; -*-

\chapter{Project Scope}
\label{cha:Project Scope}
Type extraction for existing dymanic code is still a low covered subject. Our goal is to build a complementary tool, collecting type information from an user's program and reporting this data for documentation, inspection and code migration. The reflection properties of Lua allow us to register hook functions and extract the values contained in the program's execution by accessing the values during runtime. As an output the program generates a list of function types containing all gathered information.
\paragraph*{}
Lua values can assume several types, speacially tables, which is the main data-structure mechanism of the language, and functions, considered as first class values. This type dynamism makes type inspection a challenging task, so in order to reduce some complexity, we chose to follow a merge strategy for types following the Pallene Language type specification. Pallene conventional type system brings simplicity for table types, restricting them as array types and record types and shows a straightforward function type definition. Serving as an analysis tool, we won't make any type verification or restrain the program's execution. The types that could not be infered will be shown as a dynamic type.
\paragraph*{}
The tool offers two ways for type inspection in a program:
\begin{itemize}
    \item{Full Analysis:} A full program analysis can be made by passing a Lua program as input to the extractor. In this approach, each possible function call and return types will be analysed.
    \item{Inspection library:} An auxiliar library, capable of registring specific functions for inspection. In this approach, the programmer can select what part of the program they want to analyse.
\end{itemize}
 These usage scenarios enables the extractor to be used as an auxiliary tool for migrating from dynamically to statically typed languages. At the end, a report containing information about parameters and return types of each analysed function is generated and shown to the user. This data serves as a good documentation for functions parameter and return types. Giving tools for understanding the type relations inside a program helps programmers to debug and optimize dynamically typed code.

%This document is structured as follows. In Chapter~\ref{cha:Previous Work} we present some previous work relevant to our problem. In Chapter~\ref{cha:Proposal} we explain our proposal. In Chapter~\ref{cha:Results} we show our results. Finally, in Chapter~\ref{cha:Conclusion} we present our conclusion and future work.



  % -*- coding: utf-8; -*-
\chapter{Project Specification}
\label{cha:Project Specification}
The objective of the type extractor is to generate a readable report for the user containing the types of parameter and return values of each function in a program's execution. With this objective in mind, we explored the reflection abilities of Lua through four main modules.
\par
A type function to categorize Lua values into specific types.
An inspection function to access local variable values, specially parameter and return values and pass them to our type function
A hook module to register our inspection function to be executed in each function call and return event.
Finally, a report module to print all type information gathered through the inspection.

First, we want to be able to categorize Lua values into specific types. The Type module is responsible for creating a type specification depending on the value passed.
Second, an inspection module is responsible for accessing local variables, specially paramter and return values.
Third, a hook module, responsible for registering

\paragraph*{Modules}
\begin{itemize}
    \item Type: Element that represents a type
    \item Inspect - Inspection of local upvalues
    \item Hook - Manages function hooks
    \item Report - Generates a friendly report
\end{itemize}

\paragraph*{Build}

\paragraph*{Test}

\paragraph*{Execute}



% Equation example 1:

% \begin{equation}
% \begin{split}
% \min_u \int_{x_i\in X}\int_{x_j\in X} q_{ij} u_i u_j da da + \int_{x_i\in X}||x' - x_i|| u_i da \\
% s.t. \ \ \ u\in[0,1] \ \ \land  \ \ \int_{x_i\in X}u da = a_0,
% \end{split}
% \end{equation}

% Equation exmaple 2:

% \begin{equation}
% \begin{split}
% \min_{\mathbf{u}} \alpha \mathbf{u}^T \mathbf{A}^T \mathbf{Q} \mathbf{A} \mathbf{u} +  \beta \mathbf{d}^T a' \mathbf{A} \mathbf{u} + \gamma \mathbf{u}^T \mathbf{G}^T \mathbf{G} \mathbf{u} + \delta\mathbf{f}^T a' \mathbf{A} \mathbf{u} \\
% s.t. \ \ \ \mathbf{0} \leq \mathbf{u} \leq \mathbf{1} \land \mathbf{a}^T\mathbf{u}=a_0.
% \end{split}
% \end{equation}

% Equation example 3:
% \begin{align}
% \mathbf{G}=(g_{ij}) = \left\lbrace
% \begin{array}{ll}
% \sum_{f_k\in N_f(f_i)} l_{ik} & i=j\\
% -l_{ij} & e_{ij}\in E\\
% 0 & \text{otherwise}
% \end{array}
% \right.
% \end{align}

% \lstinputlisting[label=mean,title={Mean Filter},caption={Mean Filter},language=R]{codes/mean.R}

% %% Poruguese algorithm
% %\begin{algorithm}
% %\DontPrintSemicolon
% %\Entrada{Malha e quantidade de pontos a ser amostrado}
% %\Saida{Pontos amostrados na malha}
% %\BlankLine
% %\emph{Crie um vetor de números randômicos entre $[0,1]$ com a %quantidade de pontos a ser amostrada e ordene-o}\;
% %\emph{Calcule a área total dos triângulos da malha}\;
% %\For{$i=0$ \KwTo numeroDePontos} {
% %  \emph{Navegue entre as faces acumulando a sua $\frac{area}{areaTotal}$ até achar a face com valor acumulado $\geqslant$ numerosRandomicos[i]}\;
% %  \emph{Pegue um ponto randômico dentro da face utilizando o %método de Turk e adicione no vetor do resultado}\;
% %}
% %\caption{Escolha das amostras inicias}\label{alg:sampling}
% %\end{algorithm}\DecMargin{1em}

% %% enlgish algorithm
% \begin{algorithm}
% \DontPrintSemicolon
% \KwIn{Malha e quantidade de pontos a ser amostrado}
% \KwOut{Pontos amostrados na malha}
% \BlankLine
% \emph{Crie um vetor de números randômicos entre $[0,1]$ com a quantidade de pontos a ser amostrada e ordene-o}\;
% \emph{Calcule a área total dos triângulos da malha}\;
% \For{$i=0$ \KwTo numeroDePontos} {
%   \emph{Navegue entre as faces acumulando a sua $\frac{area}{areaTotal}$ até achar a face com valor acumulado $\geqslant$ numerosRandomicos[i]}\;
%   \emph{Pegue um ponto randômico dentro da face utilizando o método de Turk e adicione no vetor do resultado}\;
% }
% \caption{Escolha das amostras inicias}\label{alg:sampling}
% \end{algorithm}\DecMargin{1em}







  % % -*- coding: utf-8; -*-

\chapter{Development}
\label{cha:Development}
The type extractor depends heavily on the Lua debug library. Our tool make use of the hook mechanism and introspective functions of the language to inspect names and values inside a program execution. In this section we will describe the implementation of our extractor, emphasizing some key parts.
\section*{Type extraction}

The core functionality of the type extractor is the creation of a sophisticated type representation supported by the \textit{type} function. A basic example of type extraction is shown in Code~\ref{type_new}.

\lstinputlisting[label=type_new,title={Type Extraction},caption={Type Extraction}, language={[5.0]Lua}]{codes/type_new.lua}

A key aspect of the type entity is the ability to generate a new representation based on the union of other two types. It's pointed in Code~\ref{type_union} how some trivial types are merged together to create a new type. In the case of the union function, we override the \textit{\_\_add} function, as we want to use the \textit{+} operator to manipulate types.
\clearpage

\lstinputlisting[label=type_union,title={Type Union},caption={Type Union}, language={[5.0]Lua}]{codes/type_union.lua}


The creation and union of types combined, enables the categorization of array types in a more elegant way. For example, we can obtain an array type by iterating over its structure, mapping each element to a new type, then reducing this array of types by folding it with our union function. The implementation of this strategy is shown in Code~\ref{array_type}.

\lstinputlisting[label=array_type,title={Array Type Creation},caption={Array Type Creation}, language={[5.0]Lua}]{codes/array_fold.lua}

\section*{Basic profiler}
The project development is inspired by a rudimentary profiler specified in Chapter 25 of the Programming in Lua book \cite{ierusalimschy_programming_2016}. This profiler gives us insight of how a function can be inspected during runtime as it can be easly expanded to explore other introspective capabilities. Code~\ref{hook} shows the hook function registered to be executed by each event. Inevitably, some functions we do not want to inspect are captured by the this hook, so we ignore these functions as soon as we identify it's not a Lua function. The counterside of this restriction is that C functions defined by the user will not be inspected.

\lstinputlisting[label=hook,title={Hook function},caption={Hook function}, language={[5.0]Lua}]{codes/hook.lua}

Notice the invocation of \textit{debug.getinfo} with a sequence of letters representing what information we want to obtain. The \textcolor{gray}{\textit{Sn}} part captures information about the function's name, location and source, useful when generating the desired report. The letter \textcolor{gray}{\textit{f}} fills the function value, which is used as a key value for several global tables shared across modules. The letter \textcolor{gray}{\textit{t}} tells us if the function event is part of a tail call. It's important to track this value when updating return values. Finally, the letter \textcolor{gray}{\textit{r}} fills information about the values being transferred, that is, parameters values in a call or return values in a return.

\section*{Local variables inspection}

\textit{Inspect} is a function imported by the Inspect module and it's implemention is shown in Code~\ref{inspect}. It's interesting to notice that the transfered values can be inspected the same way regardless of the event type. The simplification is possible due to the ntransfer and ftransfer fields obtained before. In a call event, ftransfer is always 1, it means that the index of the first transfered value is actually the index of the first parameter, while ntransfer is the number of parameters. On the other hand, in a return event the ftransfer index is not as predictable as in a call event and the ntransfer field holds the number of transfered values in the return statement.

\lstinputlisting[label=inspect,title={Inspect function},caption={Inspect function}, language={[5.0]Lua}]{codes/inspect.lua}

Another instrospective function explored by our extractor is the \textit{debug.getlocal} function which is is responsible for accessing local variables within a closure. Togheter with the transfered value indexes, \textit{getlocal} can to access the parameter and return values we want and extract it's type. This logic is shown in Code~\ref{transfered}.

\lstinputlisting[label=transfered,title={Get Transfered Types},caption={Get Transfered Types}, language={[5.0]Lua}]{codes/transfer_types.lua}

Lua functions feature a concept called \textit{proper tail calls}, meaning that the calling function does not have its respective space in the stack after a tail call is made. Lua's proper tail call design is appropriate for many programming situations, but it requires our extarctor an extra step for updating return values of functions. Because Lua does not keep any information of the calling function of a tail call, our extractor still had to keep this information for updating the return values appropriately. Code~\ref{update_return} shows that, by manipulating a stack table, we can maintain the syncronism between the return type inspection and the program's call stack.


\lstinputlisting[label=update_return,title={Update Return Types},caption={Update Return Types}, language={[5.0]Lua}]{codes/update_return.lua}

\section*{Type comparison}
In order to test if our type representation is correct, we explored the relational metamethod \textit{\_\_eq}, giving a new meaning under our type context. Two types are equal if it has the same type representation. With that in mind, an equality function can be defined recursevly, by analysing each type tag and comparing subsequent structures if needed. An example of type comparison is shown in Code~\ref{transfered}.

\lstinputlisting[label=compare,title={Type Comparison},caption={Type Comparison}, language={[5.0]Lua}]{codes/type_compare.lua}

% \paragraph*{Type}
% Categorize values into conventional types, boolean, integer, string, float, number, array, record, function
% An array type is defined by the union of the types of each element in the array. This result can be achieved by a map+reduce strategy, creating a new type for each element in the array and merging them by our union function.
% \subparagraph*{Compatibility Matrix}
% \subparagraph*{Relational Metamethods}
% \paragraph*{Inspect}
% \subparagraph*{Accessing local variables}
% iterating getlocal for each transfered value
% \paragraph*{Hook}
% \subparagraph*{Basic profiler}
% getinfo at hook events
% \paragraph*{Report}
% \subparagraph*{String formating}
% \paragraph*{Test}
% \subparagraph*{Type comparison by an equality function}


% Table example. Table~\ref{tab:res} shows results. 

% \begin{table}[!h]
% \caption{Results for devil mesh}
% \tiny
% \begin{center}
% \begin{tabular}{ m{1.1cm} m{0.95cm} m{0.95cm} m{0.95cm} m{0.95cm} m{0.95cm} m{0.95cm} m{0.95cm} m{0.95cm} m{0.95cm} } 
%  & Mean Vertex Distance & L2 Vertex Based & Mean Quadric & MSAE & L2 Normal Based & Tangential & Mean Discrete Curvature & Area Error & Volume Error\\ 
%  \hline 
% \cite{FDC03} & 0.061277 & 0.110973 & 0.236219 & 19.697900 & 0.055170 & 0.047678 & 0.090284 & 0.051443 & 0.045645 \\ 
%  \cite{JDD03} & 0.001293 & 0.002800 & 0.002289 & 21.237300 & 0.021589 & 0.013026 & 0.087991 & 0.000364 & 0.002621 \\ 
%  \cite{SRML07} & 0.001439 & 0.002880 & 0.003540 & 14.043200 & 0.012654 & 0.008911 & 0.055849 & 0.007806 & 0.000582 \\ 
%  \cite{ZFAT11} & \cellcolor{blue!25}0.000713 & \cellcolor{blue!25}0.001537 & 0.001824 & 12.171400 & \cellcolor{blue!25}0.009654 & \cellcolor{blue!25}0.005781 & \cellcolor{blue!25}0.054567 & 0.005617 & \cellcolor{blue!25}0.000425 \\ 
%  \cite{HS13} & 0.002531 & 0.004560 & 0.007108 & 13.830100 & 0.017459 & 0.010314 & 0.114528 & 0.001686 & 0.001786 \\ 
%  \cite{ZDZBL15} & 0.001623 & 0.003079 & 0.005048 & \cellcolor{blue!25}10.454200 & 0.015233 & 0.008054 & 0.094668 & 0.002629 & 0.001326 \\ 
%  \cite{YRP16} & 0.000737 & 0.001548 & \cellcolor{blue!25}0.001493 & 16.880800 & 0.014129 & 0.006974 & 0.079952 & \cellcolor{blue!25}0.000209 & 0.002375 \\ 
%  Ours & 0.000987 & 0.001902 & 0.002686 & 11.574200 & 0.010632 & 0.006796 & 0.075106 & 0.003970 & 0.000722 \\ 
%  \end{tabular}
% \end{center}
%  \label{tab:res}
% \end{table}

% \section{Comparison}
  % \chapter{Results}
\label{cha:Results}

We proposed an algorithm for triangular mesh denoising with detail preservation...

\lstinputlisting[label=mean2,title={Mean Filter},caption={Mean Filter},language=R]{codes/mean.R}
  % \chapter{Final Considerations}
\label{cha:Final Considerations}
We proposed an extraction tool for Lua programs, capable of reporting the program's function types.
% We proposed an algorithm for triangular mesh denoising with detail preservation...

% \lstinputlisting[label=mean2,title={Mean Filter},caption={Mean Filter},language=R]{codes/mean.R}
  %% ...
  \arial
  \bibliographystyle{abnt-alf} % \bibliographystyle{abnt-num}
  \bibliography{srs} 
\end{document}
