% -*- coding: utf-8; -*-

\chapter{Project Scope}
\label{cha:Project Scope}
Type extraction for dymanic code is still a low covered subject. Our goal is to build a complementary tool, collecting type information from an user's program and reporting this data for documentation, inspection and code migration. The reflection properties of Lua allow us to register hook functions and extract the values contained in the program's execution by accessing the values during runtime. As an output the program generates a list of function types containing all gathered information. Giving tools for understanding the types relations inside a program helps programmers to debug and optimize dynamically typed code.
\paragraph*{}
Lua values can assume several types, speacially tables, which is the main data-structure mechanism of the language, and functions, considered as first class values. This type dynamism makes type inspection a challenging task, so in order to reduce some complexity, we chose to follow a merge strategy for types following the Pallene Language type specification. Pallene conventional type system brings simplicity for table types, restricting them as array types and record types and shows a straightforward function type definition. Serving as an analysis tool, we won't make any type verification or restrain the program's execution. The types that could not be infered will be shown as a dynamic type.
\paragraph*{}
The extractor offers two ways to inspect functions inside a program. A full program inspection is available by passing a lua program as input to the extractor. In this approach, each Lua function will be analysed. An alternative way is to import the extractor as an auxiliar library. By importing the inspection library, the programmer can register specific functions for inspection and select what part of the program they want to analyse.

%This document is structured as follows. In Chapter~\ref{cha:Previous Work} we present some previous work relevant to our problem. In Chapter~\ref{cha:Proposal} we explain our proposal. In Chapter~\ref{cha:Results} we show our results. Finally, in Chapter~\ref{cha:Conclusion} we present our conclusion and future work.


