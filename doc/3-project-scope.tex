% -*- coding: utf-8; -*-

\chapter{Project Scope}
\label{cha:Project Scope}
Type extraction for existing dymanic code is still an uncovered subject. Our software has the purpose of collecting type information from an user's program and reporting this data for documentation, inspection and code migration. The type extractor analyse each function call and return by using reflection properties of the Lua programming language. The debug library allow us to register hook functions to inspect a program's execution, computing the types present in the code.
\paragraph*{}
The type extractor can be used by two approaches, as a full program analysis or as a inspection library.
\begin{itemize}
    \item{Full Analysis:} A full program analysis is made by passing a Lua program as input to the extractor. In this approach, each possible function call and return types will be analysed.
    \item{Inspection library:} An auxiliar library, capable of registring specific functions for inspection. In this approach, the programmer can select what part of the program they want to analyse.
\end{itemize}
In the end of each execution, a report will be generated. This report has information about parameters and return types of each analysed function. These usage scenarios enables the extractor to be used as an auxiliary tool for migrating from dynamically to statically typed languages. It also serves as a good documentation for functions parameter and return types. Giving tools for understanding the type relations inside a program helps programmers to debug and optimize dynamically typed code.

%This document is structured as follows. In Chapter~\ref{cha:Previous Work} we present some previous work relevant to our problem. In Chapter~\ref{cha:Proposal} we explain our proposal. In Chapter~\ref{cha:Results} we show our results. Finally, in Chapter~\ref{cha:Conclusion} we present our conclusion and future work.


