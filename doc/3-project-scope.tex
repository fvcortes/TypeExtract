% -*- coding: utf-8; -*-

\chapter{Project Scope}
\label{cha:Project Scope}
Type extraction for dymanic code is a challenging task. This project explores the reflective abilites of Lua to achieve the goal of building a complementary tool to collect type information from an user's program and report this data for documentation, inspection and code migration. The instrospective functions of Lua debug library allow us to inspect names and values of a running program. It also provides a hook mechanism for registering functions to trace the program's execution. As an output the program generates a list describing the function types for parameter and return values.
\par
Lua values can assume several types, speacially tables, which is the main data-structure mechanism of the language, and functions, considered as first class values. This type dynamism makes type inspection a challenging task, so in order to reduce this complexity, we chose to follow a merge strategy for types following the Pallene Language type specification. Pallene conventional type system brings simplicity for table types, restricting them as array types and record types and shows a straightforward function type definition. 
\par
(ADAPT)Serving as an analysis tool, the type extractor won't make any type validation or restrain the program's execution. The types which could not be infered will be shown as a dynamic type. 
\par
It offers two ways to inspect functions inside a program. A full program inspection is available by passing a lua program as input to the extractor. In this approach, each Lua function will be analysed. An alternative way is to import the extractor as an auxiliar library. By importing the inspection library, the programmer can register specific functions for inspection and select what part of the program they want to analyse. Giving tools for understanding the types relations inside a program helps programmers to debug and optimize dynamically typed code.



