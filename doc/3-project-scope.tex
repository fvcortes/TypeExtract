% -*- coding: utf-8; -*-

\chapter{Project Scope}
\label{cha:Project Scope}
The extractor can analyse each function call and return by using reflection properties of the Lua programming language. With the debug library, we ca\
n register hook functions to inspect the behaviour of a program's execution and then compute the types of functions and variables present in the code\
.
\newline
The Type Extractor can be used by two approaches.
\begin{itemize}
    \item{Full Analysis:} A full program analysis can be made by passing a Lua program as input to the extractor. In this approach, each possible fun\
ction call and return types will be analysed.
    \item{Inspection library:} An auxiliar library, capable of registring specific functions for inspection. In this approach, the programmer can sel\
ect what part of the program they want to analyse.
\end{itemize}
In the end of each execution, a report will be generated. This report has information about parameters and return types of each analysed function.
\newline
These usage scenarios enables the extractor to be used as an auxiliary tool for migrating from dynamically to statically typed languages. It also ser\
ves as a good documentation for functions parameter and return types. Giving tools for understanding the type relations inside a program helps progra\
mmers to debug and optimize dynamically typed code.

%This document is structured as follows. In Chapter~\ref{cha:Previous Work} we present some previous work relevant to our problem. In Chapter~\ref{cha:Proposal} we explain our proposal. In Chapter~\ref{cha:Results} we show our results. Finally, in Chapter~\ref{cha:Conclusion} we present our conclusion and future work.


